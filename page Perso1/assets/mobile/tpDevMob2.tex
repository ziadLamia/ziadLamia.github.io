\documentclass[12pt,a4paper]{article}
\usepackage[utf8]{inputenc}
\usepackage[T1]{fontenc}
\usepackage[french]{babel}
\usepackage{geometry}
\usepackage{xcolor}
\usepackage{listings}

\geometry{margin=2.2cm}

% STYLE CODE
\definecolor{kotlinBlue}{RGB}{47,91,164}
\definecolor{lightgray}{rgb}{0.95,0.95,0.95}
\lstdefinestyle{KotlinStyle}{
    backgroundcolor=\color{lightgray},
    keywordstyle=\color{kotlinBlue}\bfseries,
    basicstyle=\ttfamily\footnotesize,
    frame=single,
    breaklines=true
}

\title{\textbf{TP 2 — Listes, LazyColumn et Cartes}}
\author{Pr. Lamia ZIAD}
\date{EST Essaouira}

\begin{document}
\maketitle

\section*{Objectifs}
\begin{itemize}
    \item Manipuler les listes dans Compose ;
    \item Afficher des données avec LazyColumn ;
    \item Créer des cartes (Card) stylées ;
    \item Ajouter une interaction (clic).
\end{itemize}

\section*{Exercice 1 — Liste simple}
Créer une liste de villes :
\begin{itemize}
    \item Casablanca
    \item Marrakech
    \item Agadir
    \item Essaouira
\end{itemize}

Afficher la liste avec \texttt{LazyColumn}.

\textbf{Exemple :}
\begin{lstlisting}[style=KotlinStyle]
@Composable
fun ListeVilles() {
    val villes = listOf("Casablanca","Marrakech","Agadir","Essaouira")

    LazyColumn {
        items(villes) { ville ->
            Text(text = ville)
        }
    }
}
\end{lstlisting}

\section*{Exercice 2 — Fiches stylées (Card)}
Créer une Card pour chaque ville :
\begin{itemize}
    \item Nom de la ville ;
    \item Une icône ;
    \item Une couleur d’arrière-plan.
\end{itemize}

\section*{Exercice 3 — Interaction}
Lorsque l’utilisateur clique sur une ville, afficher un message en bas de l’écran :  
\textbf{“Vous avez choisi : [ville]”}

\textbf{Indication :} utiliser \texttt{remember { mutableStateOf(...) }}

\section*{Travail à Rendre}
\begin{itemize}
    \item Capture de la LazyColumn ;
    \item Code complet ;
    \item Explication du fonctionnement des Cards.
\end{itemize}

\end{document}
