\documentclass[12pt,a4paper]{article}
\usepackage[utf8]{inputenc}
\usepackage[T1]{fontenc}
\usepackage[french]{babel}
\usepackage{geometry}
\usepackage{xcolor}
\usepackage{enumitem}
\usepackage{listings}

\geometry{margin=2.2cm}

% ==============================
% COLORATION DU CODE KOTLIN
% ==============================
\definecolor{kotlinBlue}{RGB}{47,91,164}
\definecolor{kotlinOrange}{RGB}{255,140,0}
\definecolor{kotlinPurple}{RGB}{135,0,155}
\definecolor{lightgray}{rgb}{0.95,0.95,0.95}

\lstdefinelanguage{Kotlin}{
    morekeywords={
        class, fun, val, var, if, else, for, while, when,
        interface, override, return, println, true, false
    },
    sensitive=true,
    morecomment=[l]{//},
    morecomment=[s]{/*}{*/},
    morestring=[b]"
}

\lstdefinestyle{KotlinStyle}{
    backgroundcolor=\color{lightgray},
    keywordstyle=\color{kotlinBlue}\bfseries,
    stringstyle=\color{kotlinOrange},
    commentstyle=\color{gray},
    basicstyle=\ttfamily\footnotesize,
    frame=single,
    breaklines=true
}

% ==============================
\title{\textbf{Travaux Pratiques Kotlin \\ Variables – Collections – POO – Héritage}}
\author{Pr. Lamia ZIAD}
\date{EST Essaouira}

\begin{document}

\maketitle

\section*{I. Travaux Pratiques : Kotlin — Variables, Listes, Tableaux et Classes}

% ---------------------------------------------------------
\subsection*{Exercice 1 : Manipulation des Variables}

\textbf{Objectif :}
\begin{itemize}
    \item comprendre les types de variables en Kotlin;
    \item effectuer des opérations simples.
\end{itemize}

\textbf{Consignes :}
\begin{enumerate}
    \item Déclarer une variable \texttt{nom} de type \texttt{String} contenant votre prénom.
    \item Déclarer une variable \texttt{age} de type \texttt{Int}.
    \item Afficher : \emph{``Je m'appelle [nom] et j'ai [age] ans.''}
    \item Créer une variable \texttt{anneeNaissance} calculée automatiquement.
\end{enumerate}

\textbf{Exemple de code :}

\begin{lstlisting}[style=KotlinStyle, language=Kotlin]
val nom: String = "Ahmed"
val age: Int = 20
println("Je m'appelle $nom et j'ai $age ans.")

val anneeNaissance = 2025 - age
println("Annee de naissance : $anneeNaissance")
\end{lstlisting}

% ---------------------------------------------------------
\subsection*{Exercice 2 : Listes et Tableaux}

\textbf{Consignes :}
\begin{enumerate}
    \item Créer une liste mutable contenant trois noms de villes.
    \item Ajouter une ville à la liste.
    \item Afficher chaque ville avec une boucle.
    \item Créer un tableau contenant : 10, 20, 30, 40, 50.
    \item Calculer et afficher la somme des éléments.
\end{enumerate}

\textbf{Exemple de code :}

\begin{lstlisting}[style=KotlinStyle, language=Kotlin]
val villes = mutableListOf("Rabat", "Casablanca", "Agadir")
villes.add("Essaouira")

for (v in villes) {
    println(v)
}

val tableau = arrayOf(10, 20, 30, 40, 50)
val somme = tableau.sum()
println("Somme = $somme")
\end{lstlisting}

% ---------------------------------------------------------
\subsection*{Exercice 3 : Classes et Objets}

\textbf{Consignes :}
\begin{enumerate}
    \item Créer une classe \texttt{Personne(nom, age)}.
    \item Ajouter une méthode \texttt{sePresenter()}.
    \item Instancier deux objets et appeler la méthode.
\end{enumerate}

\textbf{Exemple de code :}

\begin{lstlisting}[style=KotlinStyle, language=Kotlin]
class Personne(val nom: String, val age: Int) {
    fun sePresenter() {
        println("Bonjour, je m'appelle $nom et j'ai $age ans.")
    }
}

val p1 = Personne("Sara", 22)
val p2 = Personne("Yassine", 30)

p1.sePresenter()
p2.sePresenter()
\end{lstlisting}

% ---------------------------------------------------------
\subsection*{Exercice 4 : Héritage et Interfaces}

\textbf{Consignes :}
\begin{itemize}
    \item Classe \texttt{Animal} avec propriété \texttt{nom} et méthode \texttt{faireDuBruit()}.
    \item Classe \texttt{Chien} : afficher ``Ouaf Ouaf!''.
    \item Classe \texttt{Chat} : afficher ``Miaou!''.
    \item Tester les deux objets.
\end{itemize}

\textbf{Exemple de code :}

\begin{lstlisting}[style=KotlinStyle, language=Kotlin]
open class Animal(val nom: String) {
    open fun faireDuBruit() {
        println("Bruit inconnu")
    }
}

class Chien(nom: String) : Animal(nom) {
    override fun faireDuBruit() {
        println("Ouaf Ouaf!")
    }
}

class Chat(nom: String) : Animal(nom) {
    override fun faireDuBruit() {
        println("Miaou!")
    }
}

val c1 = Chien("Rex")
val c2 = Chat("Mimi")

c1.faireDuBruit()
c2.faireDuBruit()
\end{lstlisting}

\newpage
\section*{II. Travaux Pratiques : Programmation Orientée Objet en Kotlin}

\subsection*{Titre : Gestion des véhicules}

\textbf{Objectifs :}
\begin{itemize}
    \item comprendre les classes et l'héritage;
    \item manipuler les interfaces;
    \item créer et utiliser des objets.
\end{itemize}

% ---------------------------------------------------------
\subsection*{Consignes — Architecture demandée}

\textbf{1. Classe Vehicule :}
\begin{itemize}
    \item propriétés : \texttt{marque}, \texttt{modele}, \texttt{vitesseMax};
    \item méthode : \texttt{afficherDetails()}.
\end{itemize}

\textbf{2. Interface Reparable :}
\begin{itemize}
    \item méthode : \texttt{reparer()}.
\end{itemize}

\textbf{3. Héritage :}
\begin{itemize}
    \item \texttt{Voiture} hérite de \texttt{Vehicule} + propriété \texttt{nombrePortes};
    \item \texttt{Moto} hérite de \texttt{Vehicule} + propriété \texttt{cylindree};
    \item les deux implémentent \texttt{Reparable}.
\end{itemize}

\textbf{4. Programme principal :}
\begin{itemize}
    \item créer une voiture et une moto;
    \item afficher les détails;
    \item simuler une réparation.
\end{itemize}

% ---------------------------------------------------------
\subsection*{Exemple de code (structure attendue)}

\begin{lstlisting}[style=KotlinStyle, language=Kotlin]
open class Vehicule(
    val marque: String,
    val modele: String,
    val vitesseMax: Int
) {
    open fun afficherDetails() {
        println("Vehicule: $marque $modele, Vitesse max: $vitesseMax km/h")
    }
}

interface Reparable {
    fun reparer()
}

class Voiture(
    marque: String,
    modele: String,
    vitesseMax: Int,
    val nombrePortes: Int
) : Vehicule(marque, modele, vitesseMax), Reparable {

    override fun afficherDetails() {
        super.afficherDetails()
        println("Nombre de portes: $nombrePortes")
    }

    override fun reparer() {
        println("La voiture $marque $modele est en reparation.")
    }
}

class Moto(
    marque: String,
    modele: String,
    vitesseMax: Int,
    val cylindree: Int
) : Vehicule(marque, modele, vitesseMax), Reparable {

    override fun afficherDetails() {
        super.afficherDetails()
        println("Cylindree: ${cylindree}cc")
    }

    override fun reparer() {
        println("La moto $marque $modele est en reparation.")
    }
}

fun main() {
    val voiture = Voiture("Toyota", "Corolla", 180, 4)
    val moto = Moto("Yamaha", "R1", 299, 1000)

    println("---- Details des vehicules ----")
    voiture.afficherDetails()
    moto.afficherDetails()

    println("\n---- Reparation ----")
    voiture.reparer()
    moto.reparer()
}
\end{lstlisting}

\end{document}
