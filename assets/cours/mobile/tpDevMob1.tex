\documentclass[12pt,a4paper]{article}
\usepackage[utf8]{inputenc}
\usepackage[T1]{fontenc}
\usepackage[french]{babel}
\usepackage{geometry}
\usepackage{listings}
\usepackage{xcolor}

\geometry{margin=2.2cm}

% STYLE CODE
\definecolor{kotlinBlue}{RGB}{47,91,164}
\definecolor{lightgray}{rgb}{0.95,0.95,0.95}
\lstdefinestyle{KotlinStyle}{
    backgroundcolor=\color{lightgray},
    keywordstyle=\color{kotlinBlue}\bfseries,
    basicstyle=\ttfamily\footnotesize,
    frame=single,
    breaklines=true
}

\title{\textbf{TP 1 — Introduction à Kotlin et Jetpack Compose}}
\author{Pr. Lamia ZIAD}
\date{EST Essaouira}

\begin{document}
\maketitle

\section*{Objectifs}
\begin{itemize}
    \item Installer Android Studio ;
    \item Créer son premier projet Compose ;
    \item Comprendre les Composables et le State ;
    \item Manipuler Text, Button, Column.
\end{itemize}

\section*{Exercice 1 — Création du premier écran}
Créer une application Jetpack Compose qui affiche un texte \textbf{“Bienvenue dans Kotlin Mobile”}.  
Utiliser un composable \texttt{@Composable fun Accueil()}.

\section*{Exercice 2 — Compteur interactif}
Ajouter :
\begin{itemize}
    \item un compteur numérique ;
    \item un bouton “Incrémenter” ;
    \item un bouton “Réinitialiser”.
\end{itemize}

\textbf{Exemple :}
\begin{lstlisting}[style=KotlinStyle]
@Composable
fun Compteur() {
    var count by remember { mutableStateOf(0) }

    Column {
        Text("Compteur = $count")
        Button(onClick = { count++ }) {
            Text("Incrementer")
        }
        Button(onClick = { count = 0 }) {
            Text("Reset")
        }
    }
}
\end{lstlisting}

\section*{Travail à Rendre}
\begin{itemize}
    \item Capture d’écran du compteur ;
    \item Code complet du composable ;
    \item Explication du fonctionnement de \texttt{remember} et \texttt{mutableStateOf}.
\end{itemize}

\end{document}
