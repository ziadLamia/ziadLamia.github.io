\documentclass[12pt,a4paper]{article}
\usepackage[utf8]{inputenc}
\usepackage[T1]{fontenc}
\usepackage[french]{babel}
\usepackage{geometry}
\usepackage{xcolor}

\geometry{margin=2.2cm}

\title{\textbf{TP 3 — MVVM, ViewModel et Navigation Compose}}
\author{Pr. Lamia ZIAD}
\date{EST Essaouira}

\begin{document}
	\maketitle
	
	\section*{Objectifs}
	\begin{itemize}
		\item Comprendre l’architecture MVVM en développement mobile ;
		\item Manipuler un ViewModel ;
		\item Utiliser Navigation Compose ;
		\item Mettre en place un écran liste + un écran détails.
	\end{itemize}
	
	% ---------------------------------------------------------
	\section*{Exercice 1 — Création d’un ViewModel}
	
	Créer un \textbf{ViewModel} contenant une liste d'utilisateurs.
	
	\textbf{Code à écrire :}
	
	\smallskip
	\texttt{
		class UserViewModel : ViewModel() \{ \\
		\quad val users = listOf("Amine", "Laila", "Yassine", "Salma") \\
		\}
	}
	
	% ---------------------------------------------------------
	\section*{Exercice 2 — Mise en place de la Navigation}
	
	Créer deux écrans :
	\begin{itemize}
		\item \textbf{ScreenListe} : affiche la liste des utilisateurs ;
		\item \textbf{ScreenDetails} : affiche ``Profil de : [nom]''.
	\end{itemize}
	
	Créer un \texttt{NavHost} avec les routes suivantes :
	\begin{itemize}
		\item \texttt{"liste"}
		\item \texttt{"details/\{nom\}"}
	\end{itemize}
	
	Exemple de navigation (à écrire dans votre code) :
	
	\smallskip
	\texttt{
		navController.navigate("details/" + nom)
	}
	
	% ---------------------------------------------------------
	\section*{Exercice 3 — Interaction Liste → Détails}
	
	Lorsque l’utilisateur clique sur un nom dans la liste, naviguer vers l’écran détails.
	
	Exemple de composable liste (à écrire vous-même) :
	\begin{itemize}
		\item une colonne ou LazyColumn ;
		\item un Text cliquable pour chaque utilisateur ;
		\item un appel à \texttt{navController.navigate(...)}.
	\end{itemize}
	
	% ---------------------------------------------------------
	\section*{Exercice 4 — Écran de profil}
	
	Créer un composable :
	
	\texttt{@Composable fun ScreenDetails(nom: String)}
	
	Affichage attendu :
	
	\bigskip
	\texttt{"Profil de : " + nom}
	
	\bigskip
	
	Vous pouvez utiliser :
	\begin{itemize}
		\item Column
		\item Text
	\end{itemize}
	
	% ---------------------------------------------------------
	\section*{Travail à Rendre}
	\begin{itemize}
		\item Code complet du ViewModel ;
		\item Code du NavHost ;
		\item Écran liste et écran détails ;
		\item Capture d’écran montrant le passage d’un écran à l’autre ;
		\item Brève explication (5 lignes) du fonctionnement de Navigation Compose.
	\end{itemize}
	
\end{document}
