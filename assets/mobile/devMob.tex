\documentclass[12pt,a4paper]{article}
\usepackage[utf8]{inputenc}
\usepackage[T1]{fontenc}
\usepackage[french]{babel}
\usepackage{geometry}
\usepackage{listings}
\usepackage{xcolor}
\usepackage{enumitem}
\usepackage{amsmath}
\usepackage{tcolorbox}
\usepackage{graphicx}

\geometry{margin=2cm}

\definecolor{ESTblue}{RGB}{0,70,140}
\definecolor{codegreen}{rgb}{0,0.6,0}
\definecolor{codegray}{rgb}{0.5,0.5,0.5}
\definecolor{codepurple}{rgb}{0.58,0,0.82}
\definecolor{backcolour}{rgb}{0.95,0.95,0.92}
\definecolor{androidgreen}{rgb}{0.32,0.67,0.28}

\lstdefinestyle{kotlinstyle}{
    backgroundcolor=\color{backcolour},
    commentstyle=\color{codegreen},
    keywordstyle=\color{magenta},
    numberstyle=\tiny\color{codegray},
    stringstyle=\color{codepurple},
    basicstyle=\ttfamily\footnotesize,
    breakatwhitespace=false,
    breaklines=true,
    captionpos=b,
    keepspaces=true,
    numbers=left,
    numbersep=5pt,
    showspaces=false,
    showstringspaces=false,
    showtabs=false,
    tabsize=2,
    frame=single,
    language=Java
}
\makeatletter
\def\maketitle{\begin{titlepage}
		\centering
		\vspace*{2cm}
		\includegraphics[width=8cm]{logo.png}\\[3cm]
		{\Huge\bfseries\textcolor{ESTblue}{DUT : Génie Informatique}}\\[0.5cm]
		{\Large\bfseries Cours de Développement Mobile avec Kotlin et Jetpack Compose}\\[0.5cm]
		{\large Pr. Lamia ZIAD}\\
		{\large École Supérieure de Technologie d'Essaouira}\\[1cm]
		\vfill
	\end{titlepage}
}
\makeatother

\begin{document}

\maketitle

\tableofcontents

\section{Introduction à Jetpack Compose}

\subsection{Qu'est-ce que Jetpack Compose ?}
Jetpack Compose est le toolkit moderne de Google pour créer des interfaces utilisateur natives sur Android. Il utilise une approche déclarative avec Kotlin.

\textbf{Avantages :}
\begin{itemize}
\item \textbf{Déclaratif} : Décrivez ce que vous voulez, pas comment le faire
\item \textbf{Concis} : Moins de code que XML
\item \textbf{Intuitif} : Code Kotlin pur
\item \textbf{Compatibilité} : Fonctionne avec les vues existantes
\end{itemize}

\subsection{Configuration de l'environnement}
\begin{lstlisting}[style=kotlinstyle]
// build.gradle.kts (Module app)
android {
    buildFeatures {
        compose = true
    }
    
    composeOptions {
        kotlinCompilerExtensionVersion = "1.5.4"
    }
}

dependencies {
    implementation("androidx.compose.ui:ui:1.5.4")
    implementation("androidx.compose.material3:material3:1.1.2")
    implementation("androidx.activity:activity-compose:1.7.2")
}
\end{lstlisting}

\section{Premiers Pas avec Compose}

\subsection{Structure de base}
\begin{lstlisting}[style=kotlinstyle]
// MainActivity.kt
class MainActivity : ComponentActivity() {
    override fun onCreate(savedInstanceState: Bundle?) {
        super.onCreate(savedInstanceState)
        setContent {
            MyAppTheme {
                Surface(
                    modifier = Modifier.fillMaxSize(),
                    color = MaterialTheme.colorScheme.background
                ) {
                    Greeting("Android")
                }
            }
        }
    }
}

@Composable
fun Greeting(name: String, modifier: Modifier = Modifier) {
    Text(
        text = "Hello $name!",
        modifier = modifier
    )
}
\end{lstlisting}

\subsection{Composants de base}
\begin{lstlisting}[style=kotlinstyle]
@Composable
fun BasicComponents() {
    Column(
        modifier = Modifier
            .fillMaxSize()
            .padding(16.dp)
    ) {
        // Texte
        Text(
            text = "Bonjour Compose!",
            style = MaterialTheme.typography.headlineMedium
        )
        
        Spacer(modifier = Modifier.height(16.dp))
        
        // Bouton
        Button(
            onClick = { /* Action */ }
        ) {
            Text("Cliquez-moi")
        }
        
        // Champ de texte
        var text by remember { mutableStateOf("") }
        TextField(
            value = text,
            onValueChange = { text = it },
            label = { Text("Entrez du texte") }
        )
    }
}
\end{lstlisting}

\section{Gestion d'État dans Compose}

\subsection{State et Recomposition}
\begin{lstlisting}[style=kotlinstyle]
@Composable
fun Counter() {
    // State avec remember
    var count by remember { mutableStateOf(0) }
    
    Column(
        horizontalAlignment = Alignment.CenterHorizontally
    ) {
        Text(
            text = "Compteur: $count",
            style = MaterialTheme.typography.headlineMedium
        )
        
        Button(
            onClick = { count++ }
        ) {
            Text("Incrementer")
        }
    }
}
\end{lstlisting}

\subsection{State Hoisting}
\begin{lstlisting}[style=kotlinstyle]
@Composable
fun CounterScreen() {
    var count by remember { mutableStateOf(0) }
    
    Counter(
        count = count,
        onIncrement = { count++ },
        onDecrement = { count-- }
    )
}

@Composable
fun Counter(
    count: Int,
    onIncrement: () -> Unit,
    onDecrement: () -> Unit
) {
    Row(
        verticalAlignment = Alignment.CenterVertically
    ) {
        Button(onClick = onDecrement) {
            Text("-")
        }
        
        Text(
            text = "$count",
            modifier = Modifier.padding(16.dp)
        )
        
        Button(onClick = onIncrement) {
            Text("+")
        }
    }
}
\end{lstlisting}

\section{Layout et Modifiers}

\subsection{Layouts principaux}
\begin{lstlisting}[style=kotlinstyle]
@Composable
fun LayoutExamples() {
    // Column - disposition verticale
    Column {
        Text("Element 1")
        Text("Element 2")
        Text("Element 3")
    }
    
    // Row - disposition horizontale
    Row {
        Text("Element A")
        Text("Element B")
        Text("Element C")
    }
    
    // Box - superposition
    Box(
        contentAlignment = Alignment.Center
    ) {
        Image(
            painter = painterResource(R.drawable.background),
            contentDescription = null
        )
        Text("Texte superpose")
    }
}
\end{lstlisting}

\subsection{Modifiers courants}
\begin{lstlisting}[style=kotlinstyle]
@Composable
fun ModifierExamples() {
    Text(
        text = "Texte stylise",
        modifier = Modifier
            .fillMaxWidth()
            .padding(16.dp)
            .background(Color.Blue)
            .clip(RoundedCornerShape(8.dp))
            .border(1.dp, Color.Black)
            .clickable { /* Click action */ }
    )
}
\end{lstlisting}

\section{Listes et Grids}

\subsection{LazyColumn - Liste défilante}
\begin{lstlisting}[style=kotlinstyle]
data class Item(val id: Int, val name: String)

@Composable
fun ItemList(items: List<Item>) {
    LazyColumn {
        items(items) { item ->
            ItemRow(item = item)
        }
    }
}

@Composable
fun ItemRow(item: Item) {
    Card(
        modifier = Modifier
            .fillMaxWidth()
            .padding(8.dp),
        elevation = CardDefaults.cardElevation(4.dp)
    ) {
        Row(
            modifier = Modifier.padding(16.dp),
            verticalAlignment = Alignment.CenterVertically
        ) {
            Text(
                text = item.name,
                style = MaterialTheme.typography.bodyLarge,
                modifier = Modifier.weight(1f)
            )
            Icon(
                imageVector = Icons.Default.ArrowForward,
                contentDescription = null
            )
        }
    }
}
\end{lstlisting}

\subsection{LazyVerticalGrid}
\begin{lstlisting}[style=kotlinstyle]
@Composable
fun PhotoGrid(photos: List<Photo>) {
    LazyVerticalGrid(
        columns = GridCells.Adaptive(minSize = 128.dp)
    ) {
        items(photos) { photo ->
            Image(
                painter = rememberImagePainter(photo.url),
                contentDescription = null,
                modifier = Modifier
                    .aspectRatio(1f)
                    .padding(4.dp)
                    .clip(RoundedCornerShape(8.dp))
            )
        }
    }
}
\end{lstlisting}

\section{Navigation avec Compose}

\subsection{Configuration de la navigation}
\begin{lstlisting}[style=kotlinstyle]
// build.gradle.kts
dependencies {
    implementation("androidx.navigation:navigation-compose:2.7.0")
}

// Navigation graph
@Composable
fun MyApp() {
    val navController = rememberNavController()
    
    NavHost(
        navController = navController,
        startDestination = "home"
    ) {
        composable("home") {
            HomeScreen(
                onNavigateToProfile = { 
                    navController.navigate("profile") 
                }
            )
        }
        composable("profile") {
            ProfileScreen(
                onBack = { navController.popBackStack() }
            )
        }
    }
}
\end{lstlisting}

\subsection{Écrans et navigation}
\begin{lstlisting}[style=kotlinstyle]
@Composable
fun HomeScreen(onNavigateToProfile: () -> Unit) {
    Column(
        modifier = Modifier
            .fillMaxSize()
            .padding(16.dp),
        horizontalAlignment = Alignment.CenterHorizontally
    ) {
        Text("Page d'accueil")
        
        Button(
            onClick = onNavigateToProfile
        ) {
            Text("Voir le profil")
        }
    }
}

@Composable
fun ProfileScreen(onBack: () -> Unit) {
    Column(
        modifier = Modifier
            .fillMaxSize()
            .padding(16.dp)
    ) {
        IconButton(onClick = onBack) {
            Icon(Icons.Default.ArrowBack, "Retour")
        }
        
        Text("Profil utilisateur")
    }
}
\end{lstlisting}

\section{Architecture MVVM avec Compose}

\subsection{ViewModel et State}
\begin{lstlisting}[style=kotlinstyle]
// ViewModel
class UserViewModel : ViewModel() {
    private val _userState = mutableStateOf(UserState())
    val userState: State<UserState> = _userState
    
    fun loadUser(userId: String) {
        viewModelScope.launch {
            _userState.value = _userState.value.copy(
                isLoading = true
            )
            
            try {
                val user = userRepository.getUser(userId)
                _userState.value = _userState.value.copy(
                    user = user,
                    isLoading = false
                )
            } catch (e: Exception) {
                _userState.value = _userState.value.copy(
                    error = e.message,
                    isLoading = false
                )
            }
        }
    }
}

data class UserState(
    val user: User? = null,
    val isLoading: Boolean = false,
    val error: String? = null
)
\end{lstlisting}

\subsection{Utilisation dans Compose}
\begin{lstlisting}[style=kotlinstyle]
@Composable
fun UserProfileScreen(
    viewModel: UserViewModel = hiltViewModel()
) {
    val state by viewModel.userState.collectAsState()
    
    when {
        state.isLoading -> {
            CircularProgressIndicator()
        }
        state.error != null -> {
            ErrorMessage(state.error!!)
        }
        state.user != null -> {
            UserDetails(state.user!!)
        }
        else -> {
            // Etat initial
        }
    }
}

@Composable
fun UserDetails(user: User) {
    Column {
        Text(text = user.name)
        Text(text = user.email)
        // ...
    }
}
\end{lstlisting}

\section{Thèmes et Material Design 3}

\subsection{Configuration du thème}
\begin{lstlisting}[style=kotlinstyle]
@Composable
fun MyAppTheme(
    darkTheme: Boolean = isSystemInDarkTheme(),
    content: @Composable () -> Unit
) {
    val colorScheme = when {
        darkTheme -> DarkColorScheme
        else -> LightColorScheme
    }
    
    MaterialTheme(
        colorScheme = colorScheme,
        typography = Typography,
        content = content
    )
}

private val LightColorScheme = lightColorScheme(
    primary = Purple40,
    secondary = PurpleGrey40,
    tertiary = Pink40
    // ... autres couleurs
)

private val DarkColorScheme = darkColorScheme(
    primary = Purple80,
    secondary = PurpleGrey80,
    tertiary = Pink80
    // ... autres couleurs
)
\end{lstlisting}

\section{Accès aux Données}

\subsection{Room Database}
\begin{lstlisting}[style=kotlinstyle]
// Entity
@Entity
data class Task(
    @PrimaryKey val id: Int,
    val title: String,
    val completed: Boolean = false
)

// DAO
@Dao
interface TaskDao {
    @Query("SELECT * FROM task")
    fun getAll(): Flow<List<Task>>
    
    @Insert
    suspend fun insert(task: Task)
    
    @Update
    suspend fun update(task: Task)
}

// Repository
class TaskRepository(private val taskDao: TaskDao) {
    fun getAllTasks(): Flow<List<Task>> = taskDao.getAll()
    
    suspend fun addTask(task: Task) = taskDao.insert(task)
}
\end{lstlisting}

\subsection{Intégration avec Compose}
\begin{lstlisting}[style=kotlinstyle]
@Composable
fun TaskListScreen(viewModel: TaskViewModel = hiltViewModel()) {
    val tasks by viewModel.tasks.collectAsState(initial = emptyList())
    
    LazyColumn {
        items(tasks) { task ->
            TaskItem(
                task = task,
                onToggle = { viewModel.toggleTask(task) }
            )
        }
    }
}

@Composable
fun TaskItem(task: Task, onToggle: () -> Unit) {
    Row(
        modifier = Modifier
            .fillMaxWidth()
            .padding(16.dp),
        verticalAlignment = Alignment.CenterVertically
    ) {
        Checkbox(
            checked = task.completed,
            onCheckedChange = { onToggle() }
        )
        
        Text(
            text = task.title,
            modifier = Modifier.weight(1f),
            textDecoration = if (task.completed) {
                TextDecoration.LineThrough
            } else {
                TextDecoration.None
            }
        )
    }
}
\end{lstlisting}

\section{Projet Pratique : Application Todo}

\subsection{Structure complète}
\begin{lstlisting}[style=kotlinstyle]
// MainActivity
class MainActivity : ComponentActivity() {
    override fun onCreate(savedInstanceState: Bundle?) {
        super.onCreate(savedInstanceState)
        setContent {
            TodoAppTheme {
                TodoApp()
            }
        }
    }
}

// Navigation
@Composable
fun TodoApp() {
    val navController = rememberNavController()
    
    NavHost(navController, "tasks") {
        composable("tasks") { 
            TaskListScreen(
                onAddTask = { 
                    navController.navigate("add_task") 
                }
            ) 
        }
        composable("add_task") { 
            AddTaskScreen(
                onBack = { navController.popBackStack() }
            ) 
        }
    }
}
\end{lstlisting}

\subsection{Écran de liste des tâches}
\begin{lstlisting}[style=kotlinstyle]
@Composable
fun TaskListScreen(
    viewModel: TaskViewModel = hiltViewModel(),
    onAddTask: () -> Unit
) {
    val tasks by viewModel.tasks.collectAsState()
    
    Scaffold(
        topBar = {
            TopAppBar(
                title = { Text("Mes Taches") }
            )
        },
        floatingActionButton = {
            FloatingActionButton(onClick = onAddTask) {
                Icon(Icons.Default.Add, "Ajouter")
            }
        }
    ) { padding ->
        if (tasks.isEmpty()) {
            EmptyState()
        } else {
            LazyColumn(modifier = Modifier.padding(padding)) {
                items(tasks) { task ->
                    TaskItem(
                        task = task,
                        onToggle = { viewModel.toggleTask(task) },
                        onDelete = { viewModel.deleteTask(task) }
                    )
                }
            }
        }
    }
}
\end{lstlisting}

\section{Testing avec Compose}

\subsection{Tests UI}
\begin{lstlisting}[style=kotlinstyle]
// Test de composant
@Test
fun shouldDisplayInitialText() {
    composeTestRule.setContent {
        MyAppTheme {
            Greeting("Test")
        }
    }
    
    composeTestRule
        .onNodeWithText("Hello Test!")
        .assertExists()
}

// Test d'interaction
@Test
fun counterShouldIncrementWhenButtonClicked() {
    composeTestRule.setContent {
        Counter()
    }
    
    composeTestRule.onNodeWithText("Compteur: 0").assertExists()
    
    composeTestRule.onNodeWithText("Incrementer").performClick()
    
    composeTestRule.onNodeWithText("Compteur: 1").assertExists()
}
\end{lstlisting}

\section{Performance et Bonnes Pratiques}

\subsection{Optimisations}
\begin{itemize}
\item \textbf{Utilisez \texttt{remember}} : Pour éviter les recalculs coûteux
\item \textbf{Évitez les lambdas instanciées dans la recomposition} : Utilisez \texttt{rememberUpdatedState}
\item \textbf{Utilisez \texttt{Lazy} lists} : Pour les longues listes
\item \textbf{Profitez du preview} : Pour développer rapidement
\end{itemize}

\subsection{Composable Previews}
\begin{lstlisting}[style=kotlinstyle]
@Preview(showBackground = true)
@Composable
fun GreetingPreview() {
    MyAppTheme {
        Greeting("Android")
    }
}

@Preview(
    showBackground = true,
    widthDp = 360,
    heightDp = 640
)
@Composable
fun TaskListPreview() {
    MyAppTheme {
        TaskListScreen(onAddTask = {})
    }
}
\end{lstlisting}

\section{Annexe : Cheatsheet Compose}

\begin{tabular}{|l|l|}
\hline
\textbf{Composant} & \textbf{Usage} \\
\hline
\texttt{Text()} & Affichage de texte \\
\texttt{Button()} & Bouton cliquable \\
\texttt{TextField()} & Champ de saisie \\
\texttt{Image()} & Affichage d'image \\
\texttt{Column()} & Layout vertical \\
\texttt{Row()} & Layout horizontal \\
\texttt{Box()} & Superposition \\
\texttt{LazyColumn()} & Liste défilante \\
\texttt{Scaffold()} & Structure d'écran \\
\hline
\end{tabular}

\begin{tabular}{|l|l|}
\hline
\textbf{Modifier} & \textbf{Effet} \\
\hline
\texttt{.fillMaxSize()} & Remplit l'espace disponible \\
\texttt{.padding()} & Ajoute de l'espace \\
\texttt{.background()} & Change la couleur de fond \\
\texttt{.clickable()} & Rend cliquable \\
\texttt{.size()} & Définit la taille \\
\texttt{.weight()} & Poids dans Row/Column \\
\hline
\end{tabular}

\end{document}